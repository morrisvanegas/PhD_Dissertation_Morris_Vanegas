\chapter{CONCLUSION} % all caps please
\label{chap:conclusion}

In this thesis we ....

\section{Pugh Chart}

Finally, we revisit the system lifecycle properties defined in Table~\ref{tab:ilities}. We use the Pugh method~\cite{Pugh1981} to qualitatively rank each of the three \ac{NIR} systems against a reference design using the ilities as the set of criteria. The reference \ac{NIR} system is a standard finger-clip-based, two-wavelength pulse oximeter. Each ility can vary between $\pm3$ indicating that the system is ranked better ($+$), worse ($-$), or the same as ($0$) the reference design. A value of 3 allows for each of the three systems to all be ranked better (or worse) than the reference design while still providing relative ranking between the three systems. The results of this ranking are shown in Table~\ref{tab:pughtable}. 

\begin{table}[]
\centering
\caption{Pugh Chart ranking of \ac{NIR} systems}
\label{tab:pughtable}
%\resizebox{\textwidth}{!}{%
\begin{tabular}{@{}lcccc@{}}
\toprule
Ility Name        & Pulse Oximeter & MOXI & MOBI & OMCI \\ \midrule
Adaptability      & 0              & 0    & 3    & 1    \\
Affordability     & 0              & 2    & -1   & -3   \\
Comfortability    & 0              & -1   & 3    & -3   \\
Conformability    & 0              & -1   & 3    & -1   \\
Extensibility     & 0              & 2    & 1    & 3    \\
Interoperability  & 0              & 0    & 3    & 2    \\
Maintainability   & 0              & 3    & -1   & -3   \\
Manufacturability & 0              & 3    & 0    & -3   \\
Modifiability     & 0              & 1    & 2    & 3    \\
Operability       & 0              & 2    & -1   & -3   \\
Portability       & 0              & 3    & -1   & -3   \\
Reconfigurability & 0              & 1    & 2    & 0    \\ \bottomrule
\end{tabular}%
%}
\end{table}

\begin{description}
   \item[Adaptability] description.
   \item[Affordability] description.
   \item[Comfortability] description.
   \item[Conformability] description.
   \item[Extensibility] Given the context, our \ac{OMCI} system be easily extended to include features from state-of-the-art \ac{DOT} research including the use of optimal wide-field illumination patterns and sizes, new \ac{SLI} illumination patterns, and compression-sensor-based tomography.  In contrast, \ac{MOBI} would leverage features from portable fNIRS systems, which rely on the use of new driving electronics and optodes, which require new circuit designs. It, however, unlike a pulse oximeter, vary the intensity of light to accomodate hair artifacts. \ac{MOXI} can do software update easily, but features to support other vital signs necesitate specific electronics that the mobile-phone in use may not have. 
   \item[Interoperability] \ac{MOXI} is not better or worse in its ability to operate with other imaging systems. By design, \ac{OMCI} is capable of being integrated in existing x-ray mammography systems. However, \ac{MOBI} receives the highest score due to the auxiliary input of the master module, allow its measurements to synchronize with any other system that can output a \ac{TTL} signal. 
   \item[Maintainability] description.
   \item[Manufacturability] description.
   \item[Modifiability] Modifiability refers to the ability to change a default set of specified parameters. Besides the color of the paper filter, \ac{MOXI} does not allow any user changes. On our \ac{MOBI} system, a use can change the source currents, detector gains, and sampling strategy (sequentialy and spatial multiplexing). \ac{OMCI} receives the highest ranking in this category due to the ability to change the wide-field and \ac{SLI} patterns, position of the \ac{RF} source location, and scaling factor sensitivity. 
   \item[Operability] description.
   \item[Portability] description.
   \item[Reconfigurability] \ac{OMCI} requires a well-aligned and calibrated system to function. In theory, \ac{MOXI} can use different colored paper filters and the Moximeter application attempts to account for the misplacement of the filter on the camera. \ac{MOBI} is by far the most reconfigurable of the three \ac{NIR} systems by simply reconnected the modules in different arrangements. However, the optode layout within a \ac{MOBI} module is fixed, which is why the rank is set to two. 
\end{description}


\textbf{Portability} refers to the ability to be easily moved. If we consider a traditional pulse oximeter as being the center of the scale, then our smartphone-based \ac{MOXI} is more portable since it only requires a piece of paper and its software can be easily downloaded. Similarly, our \ac{OMCI} breast imaging system is much less portable than a pulse oximeter due to its size and weight. \textbf{Adaptability} refers to whether a system can be used for other applications besides what it was designed for. For example, although designed for stroke recovery monitoring, the features in our \ac{MOBI} modules allow it to be used in a range of neuroimaging studies without needing re-design. \textbf{Affordability} is cost. Compared to a traditional pulse oximeter, our \ac{OMCI} system is much more expensive than our \ac{MOXI} system that only requires only a small piece of paper. The reason for not ranking three up arrows for \ac{MOXI} in affordability is because it still requires a smartphone, which a user may or may not have in their possession. \textbf{Complexity} refers to the intricacy of a system. \ac{MOXI}, \ac{MOBI}, and \ac{OMCI} increase in complexity respectively because, despite all being based on the same near-infrared imaging principle, they use an increasing number of communication protocols, units, and subsystems. \textbf{Manufacturability} refers to how difficult a system is to build. While more complex in use, our \ac{MOBI} modules have very similar optical components and electronics to a pulse oximeter. The same expertise used in designing the circuit and fabricating the physical enclosure of a pulse oximeter clip is needed for a \ac{MOBI} module. On the other hand, our \ac{OMCI} system requires not only fabricating circuits but also mechanical assemblies and sensitive optical fibers. \textbf{Operability} is similar to usability in that it quantifies how much more (or less) difficult a system is to use and operate relative to a traditional pulse oximeter. While as easy to manufacture as a finger-clip pulse oximeter, our \ac{MOBI} modules require relatively longer setup times to connect modules and affix a cap onto a user. \textbf{Scalability} attempts to quantify the difficulty in increasing the number of users. Since it only requires a piece of paper and is software-based, it is much easier and faster for a new user to obtain a \ac{MOXI} system compared to manufacturing and shipping a traditional pulse oximeter.  While the \ac{MOBI} system is as simple to manufacture as a traditional pulse oximeter, we ranked it with one up arrow due to its broad range of applications that will likely entice more users than a non-adaptable pulse oximeter. \textbf{Maintainability} is the effort and cost needed to keep a system in working condition. Our \ac{MOXI} system can be easily maintained with regular software updates and replacing its inexpensive pieces of paper. While as simple to manufacture, we ranked the \ac{MOBI} modules with one down arrow due to the higher number of components (flat-flex cables, caps, master modules) that can potentially break and require replacement. \textbf{Conformability} is the ability of the system to physically match the surface it is trying to measure. The reflectance-based design of \ac{MOXI} relies on the flat surface of a phone camera that is susceptible to motion. \ac{OMCI} is slightly more conformable than \ac{MOXI} because it has two surfaces that compress the breast preventing motion. However, the mechanical principle of compressing tissue using two fixed-shaped surfaces is identical to a traditional pulse oximeter in the sense that neither adjusts to different user shapes. The flexible-circuit-based \ac{MOBI} modules conform easily to the scalp. \textbf{Comfortability} refers to long-term use. The silicone covers and wireless capability of the \ac{MOBI} modules allow them to be used for hours at a time. On the other hand, \ac{OMCI} requires heavy compression of the breast to minimize the thickness between paddles. This is so uncomfortable that we have to limit the time in compression to less than 3 minutes. The \ac{MOXI} system, although highly portable, requires the user to actively press onto the camera phone, which can cause discomfort over long-term use compared to the passive design of a traditional finger-clip pulse oximeter. 




\section{Significance}