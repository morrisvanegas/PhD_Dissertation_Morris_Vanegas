% intro.tex:

\chapter{INTRODUCTION} % all caps please
\label{chap:introduction}

% WHY MEDICAL IMAGING?
Modern civilization has leveraged medical imaging as a fundamental clinical and research tool for years. Although x-rays dominated the field for 80 years after its invention~\cite{Gunderman2012}, the field is growing at a rapid pace due in part to the increasing availability of relatively inexpensive computational resources~\cite{Iglehart2006}. In recent decades, we have seen an emergence of new imaging technologies that improve on traditional methods being developed and commercialized, including MRI, nuclear imaging (PET, SPECT, etc.), and ultrasound~\cite{Suetens2017}. These contemporary imaging modalities have improved on the ionizing approach of x-ray, and thus, more and more frequently have taken the center stage of routine clinical use. 

Although their diagnostic capability advancements are not in question, these "contemporary medicine relies heavily on radiological and mediconuclear investigations and procedures. "Use of radiation for medical examinations and tests is the largest manmade source of radiation exposure." Nuclear imaging relies on the imaging of injected/ingested radioactive isotopes that attach to biochemically active substances in the body. Improving on these methods, ultrasound uses high-frequency sound waves to interrogate the interior of the body. Although particularly useful for imaging structures in motion, it finds a limit in dense structures like the skull and requires the application of agents on the surface. MRI is also non-ionizing, using high-energy magnets to obtain structural and functional information. It is typically the most exciting of contemporary modalities since can achieve high-resolution scans of the entire body. The drawback is that these machines are immense, extremely expensive, and require the user to be immobile during use, limiting its impact to investigations of immobile functions and to populations with the economic resources to access them. We are in need of an imaging technique that is non-invasive, non-ionizing, can be used to diagnose various areas of the body, and is portable and low-cost for use by many.

% WHY OPTICAL IMAGING. WHY NIR LIGHT?
"Optical imaging is a non-invasive and non-ionizing technology, which uses light to probe cellular and molecular function in living subjects. Visible light is a form of electromagnetic radiation, which has properties of both particles and waves. As light travels through tissue, photons can be absorbed, reflected, or scattered depending on the tissue composition. "  Though optical imaging can use agents (fluorescence and phosphorescence imaging), in this dissertation we focus on non-invasive (no agent) methods of optical imaging. Optical imaging works because ... "In addition, in the near-infrared (NIR) part of the electromagnetic spectrum, soft tissues show less scattering and absorption than in the visible band and, therefore, using NIR optical imaging enables the probing depth to be increased to a few centimeters."  The computational headwalls of the past can now be addressed. 


% WHAT IS THE BIG CHALLENGE WE ARE ADDRESSING?
My overarching goal is to demonstrate how optical imaging is a conduit for medical imaging innovations for the rest of the 21st century. To do that, we must be bold---we will address modern national and global grand challenges to show the potential breadth of application of optical imaging. The first challenge is posed by the US Agency of International Development (USAID) through their Saving Lives at Birth Initiative. The goal is to address the heightened high-risk period for babies from the onset of labor through 48 hours after birth in LMICs. This period accounts for 48 percent of maternal deaths and 54 percent of neonatal deaths annually. For the second challenge, we turn toward the brain. The Brain Research Through Advancing Innovative Neurotechnologies (BRAIN) Initiative is focused on the development and application of new technologies to image the brain for the treatment, cure, and prevention of brain disorders. Through the National Institute of Health (NIH) National Institute of Biomedical Imaging and Bioengineering (NIBIB), we will develop a portable neuroimaging system with features tailored towards use in natural environments. And finally, we will address the challenge of improving breast cancer diagnosis and prevention of unnecessary biopsies through a grant from the National Institute of Health (NIH) National Cancer Institute (NCI) for the development of an optical mammography imager that augments existing x-ray mammography systems and scans. Although the field of medical imaging is continually advancing, at the time of writing, no contemporary imaging technique is suited to address all three aforementioned challenges. 


% WHAT IS THE SCOPE. WHAT IS INCLUDED/NOT INCLUDED?
This dissertation will show the potential of optical imaging to address a variety of current application-, user-, and setting-specific needs through the development of multiple \ac{NIR} systems. Although each of the imaging systems described in this thesis will vary in attributes (such as complexity, cost, and scalability), as the title of this thesis suggests, we will focus on the following requirements:
\begin{enumerate}
  \item Each NIR system must address portable, either through a stand-alone system or through simple integration into an existing imaging modality system. 
  \item Each NIR system must be non-invasive (use no reactive agents) and non-ionizing.
  \item Each NIR system must utilize only the visible and/or near-infrared spectral window.
\end{enumerate}

To address these challenges while meeting these requirements, at times we will leverage computational improvements of light propagation models. Other times we will integrate technological advancements in sensors to improve existing techniques. We will also take a product-focused lens to ensure what we are building is addressing the needs of users (and prevent us from falling into the academic pitfall of building for the sake of building). By demonstrating use cases and designs across a variety of medical imaging attributes, we hope to show the medical community at large the benefits of non-invasive methodologies and ways to translate these technologies outside of the research setting. 

% WHAT ARE THE AIMS?
%\section{Aims and Objectives}
This thesis is separated into five aims. The first three aims refer to the development of three individual portable and/or wearable near-infrared imaging systems. We will present the design, fabrication, and characterization of these systems as well as measurements on human test subjects. The fourth aim refers to the validation of new optical systems through characterization with optical phantoms of known optical properties. Finally, the fifth aim condenses the work into a Pugh chart by comparing all three developed NIR systems to an elementary optical imaging system, a finger-clip-based pulse oximeter. 

While this introductory chapter sets the challenge and scope of the research for this dissertation, Chapter~\ref{chap:background} gives necessary background into the basics of optical imaging, details on the optical imaging techniques used in this work, and defines the ``ilities'' (attributes) that will be compared between all three systems. Chapter~\ref{chap:moxi} shows how we address the first challenge through the development of a mobile-phone-based pulse oximeter that leverages the sensors inside already ubiquitous mobile phones in LMICs. Chapter~\ref{chap:mobi} addresses the second challenge of advancing neuroimaging through the development of a wearable functional brain imaging system with features tailored towards its use in natural, unrestricted environments. The third challenge is addressed in Chapter~\ref{chap:omci}. By combining the physiological measurements from optical imaging with the structural imaging from x-ray, we not only improve stand-alone optical imaging reconstructions but also improve existing x-ray mammography, all without exposing a patient to more ionizing radiation. Chapter~\ref{chap:3dprint} discusses the use of additive manufacturing in the development of optical phantoms utilized by all three systems in the first three aims. Finally, in the conclusion in Chapter~\ref{chap:conclusion}, we compare the three systems across their ``-ilities.''


% --- EOF ---
